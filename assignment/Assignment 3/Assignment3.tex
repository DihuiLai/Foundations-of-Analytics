%%%%%%%%%%%%%%%%%%%%%%%%%%%%%%%%%%%%%%%%%
% Lachaise Assignment
% LaTeX Template
% Version 1.0 (26/6/2018)
%
% This template originates from:
% http://www.LaTeXTemplates.com
%
% Authors:
% Marion Lachaise & François Févotte
% Vel (vel@LaTeXTemplates.com)
%
% License:
% CC BY-NC-SA 3.0 (http://creativecommons.org/licenses/by-nc-sa/3.0/)
% 
%%%%%%%%%%%%%%%%%%%%%%%%%%%%%%%%%%%%%%%%%

%----------------------------------------------------------------------------------------
%	PACKAGES AND OTHER DOCUMENT CONFIGURATIONS
%----------------------------------------------------------------------------------------

\documentclass{article}
\usepackage[english]{babel}
\usepackage[utf8]{inputenc}
\usepackage{fancyhdr, graphicx}
\fancyhf{}

\rhead{\includegraphics[width=8cm]{washulogo.eps}}
\lhead{T81-574: Foundations of Analytics}

%%%%%%%%%%%%%%%%%%%%%%%%%%%%%%%%%%%%%%%%%
% Lachaise Assignment
% Structure Specification File
% Version 1.0 (26/6/2018)
%
% This template originates from:
% http://www.LaTeXTemplates.com
%
% Authors:
% Marion Lachaise & François Févotte
% Vel (vel@LaTeXTemplates.com)
%
% License:
% CC BY-NC-SA 3.0 (http://creativecommons.org/licenses/by-nc-sa/3.0/)
% 
%%%%%%%%%%%%%%%%%%%%%%%%%%%%%%%%%%%%%%%%%

%----------------------------------------------------------------------------------------
%	PACKAGES AND OTHER DOCUMENT CONFIGURATIONS
%----------------------------------------------------------------------------------------

\usepackage{amsmath,amsfonts,stmaryrd,amssymb} % Math packages

\usepackage{enumerate} % Custom item numbers for enumerations

\usepackage[ruled]{algorithm2e} % Algorithms

\usepackage[framemethod=tikz]{mdframed} % Allows defining custom boxed/framed environments

\usepackage{listings} % File listings, with syntax highlighting
\lstset{
	basicstyle=\ttfamily, % Typeset listings in monospace font
}

%----------------------------------------------------------------------------------------
%	DOCUMENT MARGINS
%----------------------------------------------------------------------------------------

\usepackage{geometry} % Required for adjusting page dimensions and margins

\geometry{
	paper=a4paper, % Paper size, change to letterpaper for US letter size
	top=2.5cm, % Top margin
	bottom=3cm, % Bottom margin
	left=2.5cm, % Left margin
	right=2.5cm, % Right margin
	headheight=14pt, % Header height
	footskip=1.5cm, % Space from the bottom margin to the baseline of the footer
	headsep=1.2cm, % Space from the top margin to the baseline of the header
	%showframe, % Uncomment to show how the type block is set on the page
}

%----------------------------------------------------------------------------------------
%	FONTS
%----------------------------------------------------------------------------------------

\usepackage[utf8]{inputenc} % Required for inputting international characters
\usepackage[T1]{fontenc} % Output font encoding for international characters

\usepackage{XCharter} % Use the XCharter fonts

%----------------------------------------------------------------------------------------
%	COMMAND LINE ENVIRONMENT
%----------------------------------------------------------------------------------------

% Usage:
% \begin{commandline}
%	\begin{verbatim}
%		$ ls
%		
%		Applications	Desktop	...
%	\end{verbatim}
% \end{commandline}

\mdfdefinestyle{commandline}{
	leftmargin=10pt,
	rightmargin=10pt,
	innerleftmargin=15pt,
	middlelinecolor=black!50!white,
	middlelinewidth=2pt,
	frametitlerule=false,
	backgroundcolor=black!5!white,
	frametitle={Command Line},
	frametitlefont={\normalfont\sffamily\color{white}\hspace{-1em}},
	frametitlebackgroundcolor=black!50!white,
	nobreak,
}

% Define a custom environment for command-line snapshots
\newenvironment{commandline}{
	\medskip
	\begin{mdframed}[style=commandline]
}{
	\end{mdframed}
	\medskip
}

%----------------------------------------------------------------------------------------
%	FILE CONTENTS ENVIRONMENT
%----------------------------------------------------------------------------------------

% Usage:
% \begin{file}[optional filename, defaults to "File"]
%	File contents, for example, with a listings environment
% \end{file}

\mdfdefinestyle{file}{
	innertopmargin=1.6\baselineskip,
	innerbottommargin=0.8\baselineskip,
	topline=false, bottomline=false,
	leftline=false, rightline=false,
	leftmargin=2cm,
	rightmargin=2cm,
	singleextra={%
		\draw[fill=black!10!white](P)++(0,-1.2em)rectangle(P-|O);
		\node[anchor=north west]
		at(P-|O){\ttfamily\mdfilename};
		%
		\def\l{3em}
		\draw(O-|P)++(-\l,0)--++(\l,\l)--(P)--(P-|O)--(O)--cycle;
		\draw(O-|P)++(-\l,0)--++(0,\l)--++(\l,0);
	},
	nobreak,
}

% Define a custom environment for file contents
\newenvironment{file}[1][File]{ % Set the default filename to "File"
	\medskip
	\newcommand{\mdfilename}{#1}
	\begin{mdframed}[style=file]
}{
	\end{mdframed}
	\medskip
}

%----------------------------------------------------------------------------------------
%	NUMBERED QUESTIONS ENVIRONMENT
%----------------------------------------------------------------------------------------

% Usage:
% \begin{question}[optional title]
%	Question contents
% \end{question}

\mdfdefinestyle{question}{
	innertopmargin=1.2\baselineskip,
	innerbottommargin=0.8\baselineskip,
	roundcorner=5pt,
	nobreak,
	singleextra={%
		\draw(P-|O)node[xshift=1em,anchor=west,fill=white,draw,rounded corners=5pt]{%
		Question \theQuestion\questionTitle};
	},
}

\newcounter{Question} % Stores the current question number that gets iterated with each new question

% Define a custom environment for numbered questions
\newenvironment{question}[1][\unskip]{
	\bigskip
	\stepcounter{Question}
	\newcommand{\questionTitle}{~#1}
	\begin{mdframed}[style=question]
}{
	\end{mdframed}
	\medskip
}

%----------------------------------------------------------------------------------------
%	WARNING TEXT ENVIRONMENT
%----------------------------------------------------------------------------------------

% Usage:
% \begin{warn}[optional title, defaults to "Warning:"]
%	Contents
% \end{warn}

\mdfdefinestyle{warning}{
	topline=false, bottomline=false,
	leftline=false, rightline=false,
	nobreak,
	singleextra={%
		\draw(P-|O)++(-0.5em,0)node(tmp1){};
		\draw(P-|O)++(0.5em,0)node(tmp2){};
		\fill[black,rotate around={45:(P-|O)}](tmp1)rectangle(tmp2);
		\node at(P-|O){\color{white}\scriptsize\bf !};
		\draw[very thick](P-|O)++(0,-1em)--(O);%--(O-|P);
	}
}

% Define a custom environment for warning text
\newenvironment{warn}[1][Warning:]{ % Set the default warning to "Warning:"
	\medskip
	\begin{mdframed}[style=warning]
		\noindent{\textbf{#1}}
}{
	\end{mdframed}
}

%----------------------------------------------------------------------------------------
%	INFORMATION ENVIRONMENT
%----------------------------------------------------------------------------------------

% Usage:
% \begin{info}[optional title, defaults to "Info:"]
% 	contents
% 	\end{info}

\mdfdefinestyle{info}{%
	topline=false, bottomline=false,
	leftline=false, rightline=false,
	nobreak,
	singleextra={%
		\fill[black](P-|O)circle[radius=0.4em];
		\node at(P-|O){\color{white}\scriptsize\bf i};
		\draw[very thick](P-|O)++(0,-0.8em)--(O);%--(O-|P);
	}
}

% Define a custom environment for information
\newenvironment{info}[1][Info:]{ % Set the default title to "Info:"
	\medskip
	\begin{mdframed}[style=info]
		\noindent{\textbf{#1}}
}{
	\end{mdframed}
}
 % Include the file specifying the document structure and custom commands

%----------------------------------------------------------------------------------------
%	ASSIGNMENT INFORMATION
%-----------------------------------the-----------------------------------------------------

%----------------------------------------------------------------------------------------
\usepackage{color}
\usepackage{listings}
\usepackage{setspace}
\definecolor{Code}{rgb}{0,0,0}
\definecolor{Decorators}{rgb}{0.5,0.5,0.5}
\definecolor{Numbers}{rgb}{0.5,0,0}
\definecolor{MatchingBrackets}{rgb}{0.25,0.5,0.5}
\definecolor{Keywords}{rgb}{0,0,1}
\definecolor{self}{rgb}{0,0,0}
\definecolor{Strings}{rgb}{0,0.63,0}
\definecolor{Comments}{rgb}{0,0.63,1}
\definecolor{Backquotes}{rgb}{0,0,0}
\definecolor{Classname}{rgb}{0,0,0}
\definecolor{FunctionName}{rgb}{0,0,0}
\definecolor{Operators}{rgb}{0,0,0}
\definecolor{Background}{rgb}{0.98,0.98,0.98}
\lstdefinelanguage{Python}{
numbers=left,
numberstyle=\footnotesize,
numbersep=1em,
xleftmargin=1em,
framextopmargin=2em,
framexbottommargin=2em,
showspaces=false,
showtabs=false,
showstringspaces=false,
frame=l,
tabsize=4,
% Basic
basicstyle=\ttfamily\small\setstretch{1},
backgroundcolor=\color{Background},
% Comments
commentstyle=\color{Comments}\slshape,
% Strings
stringstyle=\color{Strings},
morecomment=[s][\color{Strings}]{"""}{"""},
morecomment=[s][\color{Strings}]{'''}{'''},
% keywords
morekeywords={import,from,class,def,for,while,if,is,in,elif,else,not,and,or,print,break,continue,return,True,False,None,access,as,,del,except,exec,finally,global,import,lambda,pass,print,raise,try,assert},
keywordstyle={\color{Keywords}\bfseries},
% additional keywords
morekeywords={[2]@invariant,pylab,numpy,np,scipy},
keywordstyle={[2]\color{Decorators}\slshape},
emph={self},
emphstyle={\color{self}\slshape},
%
}
\title{Homework \#	3} % Title of the assignment

\begin{document}

\maketitle % Print the title
\thispagestyle{fancy}
\pagestyle{fancy}

%----------------------------------------------------------------------------------------
%	PROBLEM 1
%----------------------------------------------------------------------------------------


\section{S\&P500 index} % Unnumbered section
Your friend Joe, who has recently started a consluting firm. He is working on a project to help a client understand the S\&P500 index. Since both Joe and his client are new to the financiel industry and had little knowledge about the market, they decide to starting with public available data and do some simple analysis. Joe learned that you are working on a master's program on data analytic and called you for your opinion. Joe provide you with two data set in ".csv" format.


\begin{warn}[Data:] % Information block
\\
S\&P500 historcal index value: \textbf{"sp500indexdaily.csv"}\\
Stock price of S\&P500 listed companies (2013-2018):  \textbf{"sp500\_cmpny\_all\_stocks\_5yr.csv"}
\end{warn}

%----------------------------------------------------------------------------------------
%	PROBLEM 2
%----------------------------------------------------------------------------------------

\subsection{Data Exploration} % Numbered section

After getting the data, you decide to explore the data by visualizing it first. You started by looking at the records in \textbf{"sp500indexdaily.csv"} and made the following plots

% Subquestions numbered with letters
\begin{enumerate}[(1)]
	\item Use the "close" price of SP500 and plot it against trading dates (between 2009-01-01 and 2018-12-31). Instead of using actual date as x-axis, you can assign an integer to each trading date and set "2009-1-1" to "0", "2009-1-2" to "1",  "2009-1-3" to "2" ... etc.).
	\item In addition to the trend of SP500 against time, you look into the statistical distribution of the index. Specifically, create a histogram of the SP500 index between 2009 and 2018.
	\item Does the histogram look like any distribution that you have learned so far? Provide your thoughts on how to describe a distribution like this (hints: where are the distribution peaks; could it be superposition of multiple normal distributions)? 
\end{enumerate}

\subsection{Regression - Single Predictor}

After looking at the SP500 index change over time i.e. plots 1.1(1), you decide to build a regression model to capture the trend of SP500 index. 
\begin{enumerate}[(1)]
\item In the GLM framework that you learned, what kind of distribution will you choose from the exponential family?
\item Build the GLM model
\item How can you interpret the betas in your model, phrase a few words that you can explain to Joe?
\item Make an in sample prediction for the SP500 index between 2009-2018. Compare it with the actual SP500 index (visualize it in a graph). What does the model capture/not capture?
\item Calculate the summed-square-error (SSE) of your in sample prediction. Do one for all data points and do one for the data points between 2017-2018
\item Based on the model, where do you think the SP500 index will be by the end of year 2020.
\item In order to validate your model, you also decide to use the data between 2009-2016 as training set and use 2017-2018 as test data set. What's the summed-square-error (SSE) after applying the model on the test data set. Compare your validation SSE with the insample SSE from question (5)
\end{enumerate}

\subsection{Regression - Multiple Predictors}
After searching the website, you learned that SP500 is an index built based on the stock price 500+ companies. The largest 5 components are "Microsoft (MSFT)", "Apple Inc. (AAPL)", "Amazon.com Inc (AMZN)", "Berkshire Hathaway Inc (BRK.B)" and "Johnson \& Johnson (JNJ)". You speculate that the SP500 index might be a weighted average of the stock prices i.e. $SP500=\beta_1MSFT+\beta_2AAPL+\beta_3AMZN+\beta_4BRK.B+...+\beta0$. 

\begin{enumerate}[(1)]
\item Build a multi-predictor regression model using SP500 index as the target variable and the stock price of MSFT, AAPL, AMZN, BRK.B and JNJ as the predictors (only use the stock price and SP500 index between 2013-02-08 and 2018-02-07). Hint: you need to reformat the company data so that each column represents the stock price and the row represents the date. A sample code is provided to you.
\item What is the in-sample prediction summed-square-error compare (SSE) now, is it better than what you get in 1.2(5) on average?
\item Visualize your prediction and compare it with the true value of SP500?
\item Check the significance of the variables by looking at the p-values of each variable, are the coefficients significant? How can you phrase a few words to explain the coefficients to Joe?
\item Can you use the model to predict future SP500 price, what is preventing you from doing that?
\end{enumerate}

\section{Mammal Classification Tree}
You were given a data set \textbf{"zoo.csv"} that includes 101 animals and a list of characteristics of the animals e.g. do they have feather, do they lay eggs or not etc. Build a CART model to classify if an animal is mammal or not.

\begin{enumerate}[(1)]
\item Calculate the overall entropy of of the target variable "ismammal", using the definition $H=p_1 log(p_1)+(1-p_1) log(1-p_1)$
\item To build a classification tree, you need to decide the splitter for each nodes of a binary tree. Using the criterion that $hair>0.5$ and split the dataset in to two branches. Calculate the entropy at each branch and the average entropy change.
\item Check the entropy changes for all the following features i.e. 'feathers', 'eggs', 'airborne', 'aquatic' and 'backbone'. Which one would you use to make the first split?
\item Build a CART model using the sklearn package and compare the model with your calculation. Is the first split the same as yours? You may use the python code provided in "CARTmammals.py"

\end{enumerate}

\end{document}
