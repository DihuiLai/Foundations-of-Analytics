%%%%%%%%%%%%%%%%%%%%%%%%%%%%%%%%%%%%%%%%%
% Lachaise Assignment
% LaTeX Template
% Version 1.0 (26/6/2018)
%
% This template originates from:
% http://www.LaTeXTemplates.com
%
% Authors:
% Marion Lachaise & François Févotte
% Vel (vel@LaTeXTemplates.com)
%
% License:
% CC BY-NC-SA 3.0 (http://creativecommons.org/licenses/by-nc-sa/3.0/)
% 
%%%%%%%%%%%%%%%%%%%%%%%%%%%%%%%%%%%%%%%%%

%----------------------------------------------------------------------------------------
%	PACKAGES AND OTHER DOCUMENT CONFIGURATIONS
%----------------------------------------------------------------------------------------


\documentclass{article}
\usepackage[english]{babel}
\usepackage[utf8]{inputenc}
\usepackage{fancyhdr, graphicx}
\fancyhf{}

\rhead{\includegraphics[width=8cm]{washulogo.eps}}
\lhead{T81-574: Foundations of Analytics}

\input{structure.tex} % Include the file specifying the document structure and custom commands

%----------------------------------------------------------------------------------------
%	ASSIGNMENT INFORMATION
%-----------------------------------the-----------------------------------------------------

%----------------------------------------------------------------------------------------
\usepackage{color}
\usepackage{listings}
\usepackage{setspace}
\definecolor{Code}{rgb}{0,0,0}
\definecolor{Decorators}{rgb}{0.5,0.5,0.5}
\definecolor{Numbers}{rgb}{0.5,0,0}
\definecolor{MatchingBrackets}{rgb}{0.25,0.5,0.5}
\definecolor{Keywords}{rgb}{0,0,1}
\definecolor{self}{rgb}{0,0,0}
\definecolor{Strings}{rgb}{0,0.63,0}
\definecolor{Comments}{rgb}{0,0.63,1}
\definecolor{Backquotes}{rgb}{0,0,0}
\definecolor{Classname}{rgb}{0,0,0}
\definecolor{FunctionName}{rgb}{0,0,0}
\definecolor{Operators}{rgb}{0,0,0}
\definecolor{Background}{rgb}{0.98,0.98,0.98}
\lstdefinelanguage{Python}{
numbers=left,
numberstyle=\footnotesize,
numbersep=1em,
xleftmargin=1em,
framextopmargin=2em,
framexbottommargin=2em,
showspaces=false,
showtabs=false,
showstringspaces=false,
frame=l,
tabsize=4,
% Basic
basicstyle=\ttfamily\small\setstretch{1},
backgroundcolor=\color{Background},
% Comments
commentstyle=\color{Comments}\slshape,
% Strings
stringstyle=\color{Strings},
morecomment=[s][\color{Strings}]{"""}{"""},
morecomment=[s][\color{Strings}]{'''}{'''},
% keywords
morekeywords={import,from,class,def,for,while,if,is,in,elif,else,not,and,or,print,break,continue,return,True,False,None,access,as,,del,except,exec,finally,global,import,lambda,pass,print,raise,try,assert},
keywordstyle={\color{Keywords}\bfseries},
% additional keywords
morekeywords={[2]@invariant,pylab,numpy,np,scipy},
keywordstyle={[2]\color{Decorators}\slshape},
emph={self},
emphstyle={\color{self}\slshape},
%
}
\title{Homework \# 1} % Title of the assignment

\begin{document}

\maketitle % Print the title
\thispagestyle{fancy}
\pagestyle{fancy}
%----------------------------------------------------------------------------------------
%	PROBLEM 1
%----------------------------------------------------------------------------------------

\section{Linear Algebra in Numpy}

\begin{enumerate}[(1)]
\item Create a random 100-by-100 matrix M, using numpy method "np.random.randn(100, 100)", where each element is drawn from a random normal distribution.
\item Calculate the mean and variance of all the elements in M; 
\item Use "for loop" to calculate the mean and variance of each row of M.
\item Use matrix operation instead of "for loop" to calculate the mean of each row of M, hint: create a vector of ones using np.ones(100, 1)?  
\item Calculate the inverse matrix $M^{-1}$
\item Verify that $M^{-1}M = MM^{-1}= I$. Are the off-diagnoal elements exactly 0, why?

\end{enumerate}


\section{Probability Distribution}
You have recently joined a data science team and working on a project that needs to simulate 5 types of distributions (Bernoulli, Poisson, Gaussian, uniform and Rolling-Dice distribution). Your teammate provides you with a simulated data sample \textbf{"sample\_trials.csv"}. In the file, each column contains 5000 numbers drawn from one of the 5 distributions. However, the columns are not labeled properly and you have to figure out the labels yourself as your teammate is on vacation.

\begin{enumerate}[(1)]
\item Do the columns have discrete value or continuous value? How many unique values does each column have? 
\item What are the min, max, mean, variance of the columns?
\item Investigate the distribution of each column by plotting the histograms. Make sure you choose the bin size properly.
\item Based on the analysis above, label each column with the distribution name and explain why. 
\item Knowing the mean, variance, write down the formulas of the distributions.
\item Simulate another 5000 samples of Bernoulli distribution with the same set of parameters. Write it into a text file.
\end{enumerate}

\end{document}
