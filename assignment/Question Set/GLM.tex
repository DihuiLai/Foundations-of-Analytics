%%%%%%%%%%%%%%%%%%%%%%%%%%%%%%%%%%%%%%%%%
% Lachaise Assignment
% LaTeX Template
% Version 1.0 (26/6/2018)
%
% This template originates from:
% http://www.LaTeXTemplates.com
%
% Authors:
% Marion Lachaise & François Févotte
% Vel (vel@LaTeXTemplates.com)
%
% License:
% CC BY-NC-SA 3.0 (http://creativecommons.org/licenses/by-nc-sa/3.0/)
% 
%%%%%%%%%%%%%%%%%%%%%%%%%%%%%%%%%%%%%%%%%

%----------------------------------------------------------------------------------------
%	PACKAGES AND OTHER DOCUMENT CONFIGURATIONS
%----------------------------------------------------------------------------------------


\documentclass{article}
\usepackage[english]{babel}
\usepackage[utf8]{inputenc}
\usepackage{fancyhdr, graphicx}
\fancyhf{}

\rhead{\includegraphics[width=8cm]{washulogo.eps}}
\lhead{T81-574: Foundations of Analytics}

\input{structure.tex} % Include the file specifying the document structure and custom commands

%----------------------------------------------------------------------------------------
%	ASSIGNMENT INFORMATION
%-----------------------------------the-----------------------------------------------------

%----------------------------------------------------------------------------------------
\usepackage{color}
\usepackage{listings}
\usepackage{setspace}
\definecolor{Code}{rgb}{0,0,0}
\definecolor{Decorators}{rgb}{0.5,0.5,0.5}
\definecolor{Numbers}{rgb}{0.5,0,0}
\definecolor{MatchingBrackets}{rgb}{0.25,0.5,0.5}
\definecolor{Keywords}{rgb}{0,0,1}
\definecolor{self}{rgb}{0,0,0}
\definecolor{Strings}{rgb}{0,0.63,0}
\definecolor{Comments}{rgb}{0,0.63,1}
\definecolor{Backquotes}{rgb}{0,0,0}
\definecolor{Classname}{rgb}{0,0,0}
\definecolor{FunctionName}{rgb}{0,0,0}
\definecolor{Operators}{rgb}{0,0,0}
\definecolor{Background}{rgb}{0.98,0.98,0.98}
\lstdefinelanguage{Python}{
numbers=left,
numberstyle=\footnotesize,
numbersep=1em,
xleftmargin=1em,
framextopmargin=2em,
framexbottommargin=2em,
showspaces=false,
showtabs=false,
showstringspaces=false,
frame=l,
tabsize=4,
% Basic
basicstyle=\ttfamily\small\setstretch{1},
backgroundcolor=\color{Background},
% Comments
commentstyle=\color{Comments}\slshape,
% Strings
stringstyle=\color{Strings},
morecomment=[s][\color{Strings}]{"""}{"""},
morecomment=[s][\color{Strings}]{'''}{'''},
% keywords
morekeywords={import,from,class,def,for,while,if,is,in,elif,else,not,and,or,print,break,continue,return,True,False,None,access,as,,del,except,exec,finally,global,import,lambda,pass,print,raise,try,assert},
keywordstyle={\color{Keywords}\bfseries},
% additional keywords
morekeywords={[2]@invariant,pylab,numpy,np,scipy},
keywordstyle={[2]\color{Decorators}\slshape},
emph={self},
emphstyle={\color{self}\slshape},
%
}
\title{Problem Set - Generalized Linear Model} % Title of the assignment

\begin{document}

\maketitle % Print the title
\thispagestyle{fancy}
\pagestyle{fancy}
%----------------------------------------------------------------------------------------
%	PROBLEM 1
%----------------------------------------------------------------------------------------

\section{Exponential Family}
The exponential family has PDF function 
\begin{align*}
p(y, \theta, \phi)=\exp\left(\frac{y\theta-b(\theta)}{a(\phi)}+c(y, \phi)\right)
\end{align*}
when 
\begin{align*}
b(\theta)&=e^\theta, \\
a(\phi)&=1, \\
c(y, \phi)&=-\log(y!)
\end{align*}
Prove that $p(y, \theta, \phi)$ is a Poisson distribution. Hint: re-parameterize $\theta=\log(\lambda)$

\section{GLM}
Given a data set of $n$ data points $[y_i, \vec{x}_i]$, $i=1, 2, 3, ..., n$. Assume that $\phi$ is constant and $\theta$ is dependent on $\vec{x}_i$, the log-likelihood function have a format $\ell=\sum\limits_{i=1}^n [\frac{y_i\theta_i-b(\theta_i)}{a(\phi)}+c(y_i, \phi)]$. To maximize the $\ell$ is equivalent to maximize 
$\ell=\sum\limits_{i=1}^n [y\theta_i-b(\theta_i)]$. In GLM, we model $\theta_i$ as a function of $\vec{x_i}\cdot\vec{\beta}$ i.e. $b'(\theta_i)=g(\vec{x_i}\cdot\vec{\beta})$
\begin{enumerate}
\item Prove that the $jth$ component of the gradient  $\nabla\ell_j=\sum\limits_{i=1}^n[y_i-b'(\theta_i)]\frac{\partial\theta_i}{\partial\beta_j}$
\item Prove that $\frac{\partial \theta_i}{\partial \beta_j}=\frac{g'(\vec{x}_i\cdot\beta)}{b''(\theta_i)}x_{ij}$, here $x_{ij}$ is the $jth$ component of vector $\vec{x}_i$. Hint: calculate the partial derivative of the equation $b'(\theta_i)=g(\vec{x_i}\cdot\vec{\beta})$ against $\beta_j$ on both side.
\end{enumerate}
\end{document}
