%%%%%%%%%%%%%%%%%%%%%%%%%%%%%%%%%%%%%%%%%
% Lachaise Assignment
% LaTeX Template
% Version 1.0 (26/6/2018)
%
% This template originates from:
% http://www.LaTeXTemplates.com
%
% Authors:
% Marion Lachaise & François Févotte
% Vel (vel@LaTeXTemplates.com)
%
% License:
% CC BY-NC-SA 3.0 (http://creativecommons.org/licenses/by-nc-sa/3.0/)
% 
%%%%%%%%%%%%%%%%%%%%%%%%%%%%%%%%%%%%%%%%%

%----------------------------------------------------------------------------------------
%	PACKAGES AND OTHER DOCUMENT CONFIGURATIONS
%----------------------------------------------------------------------------------------


\documentclass{article}
\usepackage[english]{babel}
\usepackage[utf8]{inputenc}
\usepackage{fancyhdr, graphicx}
\fancyhf{}

\rhead{\includegraphics[width=8cm]{washulogo.eps}}
\lhead{T81-574: Foundations of Analytics}

\input{structure.tex} % Include the file specifying the document structure and custom commands

%----------------------------------------------------------------------------------------
%	ASSIGNMENT INFORMATION
%-----------------------------------the-----------------------------------------------------

%----------------------------------------------------------------------------------------
\usepackage{color}
\usepackage{listings}
\usepackage{setspace}
\definecolor{Code}{rgb}{0,0,0}
\definecolor{Decorators}{rgb}{0.5,0.5,0.5}
\definecolor{Numbers}{rgb}{0.5,0,0}
\definecolor{MatchingBrackets}{rgb}{0.25,0.5,0.5}
\definecolor{Keywords}{rgb}{0,0,1}
\definecolor{self}{rgb}{0,0,0}
\definecolor{Strings}{rgb}{0,0.63,0}
\definecolor{Comments}{rgb}{0,0.63,1}
\definecolor{Backquotes}{rgb}{0,0,0}
\definecolor{Classname}{rgb}{0,0,0}
\definecolor{FunctionName}{rgb}{0,0,0}
\definecolor{Operators}{rgb}{0,0,0}
\definecolor{Background}{rgb}{0.98,0.98,0.98}
\lstdefinelanguage{Python}{
numbers=left,
numberstyle=\footnotesize,
numbersep=1em,
xleftmargin=1em,
framextopmargin=2em,
framexbottommargin=2em,
showspaces=false,
showtabs=false,
showstringspaces=false,
frame=l,
tabsize=4,
% Basic
basicstyle=\ttfamily\small\setstretch{1},
backgroundcolor=\color{Background},
% Comments
commentstyle=\color{Comments}\slshape,
% Strings
stringstyle=\color{Strings},
morecomment=[s][\color{Strings}]{"""}{"""},
morecomment=[s][\color{Strings}]{'''}{'''},
% keywords
morekeywords={import,from,class,def,for,while,if,is,in,elif,else,not,and,or,print,break,continue,return,True,False,None,access,as,,del,except,exec,finally,global,import,lambda,pass,print,raise,try,assert},
keywordstyle={\color{Keywords}\bfseries},
% additional keywords
morekeywords={[2]@invariant,pylab,numpy,np,scipy},
keywordstyle={[2]\color{Decorators}\slshape},
emph={self},
emphstyle={\color{self}\slshape},
%
}
\title{Problems} % Title of the assignment

\begin{document}

\maketitle % Print the title
\thispagestyle{fancy}
\pagestyle{fancy}
%----------------------------------------------------------------------------------------
%	PROBLEM 1
%----------------------------------------------------------------------------------------

\section{Maximum Likelihood}
\subsection{Problem}

Given a Gaussian distribution $f(y,\mu, \sigma=1)=\frac{1}{\sqrt{2\pi}}\exp(-\frac{(y-\mu)^2}{2})$ and a set of observations $y_1$, $y_2$ ..., $y_n$

\begin{enumerate}
\item  if we only have one observation $y_1=1$, what is the $\mu$ that will maximize the likelihood of the observation?
\item  if $y_1=0$, what is the $\mu$ that will maximize the likelihood of the observation?
\item  if we have observed both data points $y_1=0$ and $y_2=0$, what is the $\mu$ that will maximize the likelihood of the observations?
\end{enumerate}

\subsection{Solution}
The likelihood function of the problem, $L=\frac{1}{(\sqrt{2\pi})^n}\exp\left( -\sum\limits_{i=1}^n\frac{(y_i-\mu)^2}{2}\right)$ or the loglikelihood function $\ell=-\sum\limits_{i=1}^n\frac{(y_i-\mu)^2}{2}+const$
\begin{enumerate}
\item when $y_1=1$, $\ell(y)=\frac{(y-\mu)^2}{2}+const$ is maximized if $\mu=1$ as it gives $-\sum\limits_{i=1}^n\frac{(1-\mu)^2}{2}=0$, any $\mu$ that is different from $\mu=1$ will leads to a negative first term.
\item when $y_1=0$, $\ell(y)$ is maximized if $\mu=0$ as it gives $-\sum\limits_{i=1}^n\frac{(0-\mu)^2}{2}=0$, any $\mu$ that is different from $\mu=0$ will leads to a negative first term.
\item  if we have two observations $y_1=0$ and $y_2=0$, the corresponding loglikelihood function is $\ell(y_1=1,y_2=0)=\frac{(1-\mu)^2}{2}+\frac{(0-\mu)^2}{2}$, the $\mu$ that maximize the function satisfy the condtion $\frac{d\ell}{d\mu}=0$, which is 
$$(1-\mu)+(0-\mu)=0$$ and therefore $\mu=\frac{1+0}{2}=0.5$
\end{enumerate}
\end{document}
